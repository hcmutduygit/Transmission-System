\chapter{Chọn bộ truyền đai}
\section{Chọn loại đai}
Dựa vào công suất động cơ là $P_{dc} = 4.275 \, \text{kW}$ và số vòng quay $n_{dc} = 1450$ vòng/phút \\
$\Rightarrow$ Chọn đai loại A \\    
\begin{table}[H]
\begin{tabular}{|c|c|c|c|c|c|c|c|}
    \hline 
    Ký hiệu đai & $b_p$  & $b_0$  & h  & $y_0$ (mm) & A ($mm^2$) & Chiều dài đai (m) & $d_{1min} (mm)$ \\ \hline
    A & 11 & 13 & 8 & 2.8 & 81 & 560 $\div$ 4000 & 90 \\ \hline
\end{tabular}
\caption{Thông số đai loại A}
\end{table}
\section{Tính đường kính bánh đai nhỏ}
\begin{equation}
    d_1 = 1.2d_{min} = 1.2 \cdot 90 = 108 \, \text{(mm)}
\end{equation}
Chọn theo dãy giá trị tiêu chuẩn, ta chọn $d_1 = 112 \, \text{(mm)}$.\\
Vận tốc dài trên bánh đai nhỏ:
\begin{equation}
    v_1 = \frac{\pi d_1 n_1}{60000} = \frac{\pi \cdot 112 \cdot 1450}{60000} = 8.503 \, \text{(m/s)}
\end{equation}
$\Rightarrow$ Thỏa điều kiện $v_1 < 25 \, \text{(m/s)}$ \\

\section{Chọn hệ số trượt tương đối và tính đường kính bánh đai lớn}
Chọn hệ số trượt tương đối $\xi = 0.01$. Từ công thức tỉ số của bộ truyền đai:
\begin{equation}
    u_d = \frac{d_2}{d_1 (1 - \xi)}
\end{equation}
\begin{equation}
    d_2 = u_d \cdot d_1 \cdot (1 - \xi) = 2.44 \cdot 112 \cdot (1 - 0.01) = 270.55 \, \text{(mm)}
\end{equation}
Chọn theo dãy giá trị tiêu chuẩn, ta chọn $d_2 = 280 \, \text{(mm)}$.\\
Tính lại tỷ số truyền:
\begin{equation}
    u_d = \frac{d_2}{d_1 (1 - \xi)} = \frac{280}{112 \cdot 0.99} = 2.52
\end{equation}
Để sai số tỷ số truyền bằng 0, ta tính lại đường kính bánh đai nhỏ:
\begin{equation}
    d_1 = \frac{d_2}{u_d \cdot (1 - \xi)} = \frac{280}{2.44 \cdot (1 - 0.01)} = 115.91 \, \text{(mm)}
\end{equation}

\section{Chọn khoảng cách trục $a$}
Theo các thông số $u_d = 2.44$ và $d_2 = 280 \, \text{mm}$:
\begin{equation}
    a = 1.2d_2 = 1.2 \cdot 280 = 336 \, \text{(mm)}
\end{equation}
Chiều dài đai:
\begin{equation}
    L = 2a + \pi \frac{d_1 + d_2}{2} + \frac{(d_2 - d_1)^2}{4a}
\end{equation}
\[
    L = 2 \cdot 336 + \pi \frac{115.91 + 280}{2} + \frac{(280 - 115.91)^2}{4 \cdot 336} = 1313.93 \, \text{(mm)}
\]
$\Rightarrow$ Chọn chiều dài đai $L = 1400 \, \text{mm}$ theo dãy giá trị tiêu chuẩn.\\
Tính lại khoảng cách trục:
\begin{equation}
    k = L - \pi \frac{d_1 + d_2}{2} = 1400 - \pi \frac{115.91 + 280}{2} = 778.106 \, \text{(mm)}
\end{equation}
\begin{equation}
    \Delta = \frac{d_2 - d_1}{2} = \frac{280 - 115.91}{2} = 82.045 \, \text{(mm)}
\end{equation}
\begin{equation}
    a = \frac{k + \sqrt{k^2 - 8\Delta^2}}{4} = \frac{778.106 + \sqrt{778.106^2 - 8 \cdot 82.045^2}}{4} = 380.2 \, \text{(mm)}
\end{equation}
Kiểm tra $a$ thỏa điều kiện:
\begin{equation}
    2(d_1 + d_2) \geq a \geq 0.7(d_1 + d_2)
\end{equation}
\[
    2(115.91 + 280) \geq a \geq 0.7(115.91 + 280)
\]
\[
    791.82 \geq a \geq 277.173
\]
$\Rightarrow a = 380.2 \, \text{(mm)}$ thỏa điều kiện. \\

\section{Tính toán vận tốc đai và số truyền đai}
Vận tốc dây đai:
\begin{equation}
    v = \frac{\pi d_1 n_1}{60000} = \frac{\pi \cdot 115.91 \cdot 1450}{60000} = 8.8 \, \text{(m/s)}
\end{equation}
Số vòng chạy đai trong 1 giây:
\begin{equation}
    i = \frac{v_1}{L} = \frac{8.8}{1400 \cdot 10^{-3}} = 6.286 \, \text{s}^{-1}
\end{equation}
$\Rightarrow$ Thỏa điều kiện $i \leq [i] = 10 \, \text{s}^{-1}$. \\

\section{Tính góc ôm đai bánh nhỏ}
\begin{equation}
    \alpha_1 = 180 - 57 \frac{d_2 - d_1}{a} = 180 - 57 \frac{280 - 115.91}{380.2} = 155.29^{\circ}
\end{equation}
\section{Các hệ số sử dụng}
Hệ số xét đến ảnh hưởng góc ôm đai:
\begin{equation}
    C_{\alpha} = 1.24(1 - e^{\frac{-\alpha_1}{110}}) = 1.24(1 - e^{\frac{-155.29}{110}}) = 0.938
\end{equation}

Hệ số xét đến ảnh hưởng vận tốc:
\begin{equation}
    C_v = 1 - 0.05(0.01v^2 - 1) = 1 - 0.05(0.01 \cdot 8.8^2 - 1) = 1.011
\end{equation}

Hệ số xét đến ảnh hưởng tỷ số truyền $u$:
\begin{equation}
    C_u = 1.14 \quad (v > 2.5 \, \text{m/s})
\end{equation}

Hệ số xét đến ảnh hưởng của chiều dài đai $L$:
\begin{equation}
    C_L = \sqrt[6]{\frac{L}{L_0}} = \sqrt[6]{\frac{1400}{1700}} = 0.968
\end{equation}

Hệ số xét đến sự ảnh hưởng của sự phân bố không đều tải trọng giữa các dây đai:
\begin{equation}
    C_z = 0.95 \quad \text{(chọn sơ bộ)}
\end{equation}

Hệ số xét đến ảnh hưởng của chế độ tải trọng:
\begin{equation}
    C_r = 0.9 \quad \text{(chọn sơ bộ)}
\end{equation}

Chọn công suất có ích cho phép theo GOST 1284.3 - 96, ta có: \\
Đai loại A, $d_1 = 115.91 \, \text{mm}$, $v_1 = 8.8 \, \text{m/s}$ \\
$\Rightarrow$ Chọn $[P_0] = 1.80$.

Tính số dây đai theo công thức:
\begin{equation}
    z \geq \frac{P_1}{[P_0] \cdot C_{\alpha} \cdot C_u \cdot C_L \cdot C_z \cdot C_r \cdot C_v} = \frac{4.102}{1.8 \cdot 0.938 \cdot 1.14 \cdot 0.968 \cdot 0.95 \cdot 0.9 \cdot 1.011} = 2.55
\end{equation}
$\Rightarrow$ Chọn $z = 3$ dây đai. \\
Kiểm nghiệm lại $C_z$: vì $z = 3$ nên $C_z = 0.95$ như đã chọn sơ bộ.

\section{Lực trên dây đai}
Tổng lực căng đai ban đầu trên cả dây đai:
\begin{equation}
    F_0 = z \cdot A \cdot [\sigma_0] = 3 \cdot 81 \cdot 1 = 243 \, \text{(N)}
\end{equation}
Trong đó: Đối với đai thang, $\sigma_0 \leq 1.5 \, \text{MPa}$ nên ta chọn $\sigma_0 = 1 \, \text{MPa}$, $z = 3$, $A = 81 \, \text{mm}^2$.

Lực căng trên mỗi dây đai:
\begin{equation}
    \frac{F_0}{z} = \frac{243}{3} = 81 \, \text{(N)}
\end{equation}

Tổng lực vòng có ích trên cả 3 đai:
\begin{equation}
    F_t = \frac{1000P_1}{v_1} = \frac{1000 \cdot 4.102}{8.8} = 466.136 \, \text{(N)}
\end{equation}

Lực vòng có ích trên mỗi dây đai:
\begin{equation}
    \frac{F_t}{z} = \frac{466.136}{3} = 155.379 \, \text{(N)}
\end{equation}

\section{Lực tác dụng lên trục}
\begin{equation}
    F_r \approx 2F_0\sin\left(\frac{\alpha_1}{2}\right) = 2 \cdot 243 \sin\left(\frac{155.29}{2}\right) = 474.744 \, \text{(N)}
\end{equation}

\section{Ứng suất lớn nhất trong dây đai}
\begin{equation}
    \sigma_{max} = \sigma_1 + \sigma_v + \sigma_{F1} = \sigma_0 + 0.5\sigma_t + \sigma_v + \sigma_{F1}
\end{equation}
\begin{equation}
    = \frac{F_0}{A} + 0.5 \cdot \frac{F_t}{A} + \rho \cdot v^2 \cdot 10^{-6} + E \cdot \frac{2 \cdot y_0}{d_1}
\end{equation}
\[
    = \frac{243}{3 \cdot 81} + 0.5 \cdot \frac{466.136}{3 \cdot 81} + 1000 \cdot 8.8^2 \cdot 10^{-6} + 60 \cdot \frac{2 \cdot 2.8}{115.19} = 4.88 \, \text{(MPa)}
\]

\section{Tuổi thọ dây đai}
\begin{equation}
    L_h = \frac{\left(\frac{\sigma_r}{\sigma_{max}}\right)^m \cdot 10^7}{2 \cdot 3600 \cdot i} = \frac{\left(\frac{9}{6.9}\right)^8 \cdot 10^7}{2 \cdot 3600 \cdot 6.286} = 29571.59 \, \text{(h)}
\end{equation}
Trong đó:
\begin{itemize}
    \item $\sigma_r = 9 \, \text{(MPa)}$ - giới hạn mỏi của đai thang.
    \item $m = 8$ - chỉ số mũ của đường cong mỏi đối với đai thang.
    \item $i = 6.286 \, \text{(s}^{-1}\text{)}$ - số vòng chạy của đai trong một giây.
\end{itemize}
\section{Bảng thông số bộ truyền đai}
\begin{table}[H]
    \centering
    \begin{tabular}{|l|c|}
        \hline
        \textbf{Thông số} & \textbf{Giá trị} \\ \hline
        Loại đai & A \\ \hline
        Đường kính bánh dẫn, $d_1$ (mm) & 115.91 \\ \hline
        Đường kính bánh bị dẫn, $d_2$ (mm) & 280 \\ \hline
        Chiều dài dây đai, $L$ (mm) & 1400 \\ \hline
        Khoảng cách trục, $a$ (mm) & 380.2 \\ \hline
        Góc ôm đai, $\alpha_1$ ($^\circ$) & 155.29 \\ \hline
        Số dây đai, $z$ & 3 \\ \hline
        Tuổi thọ đai, $L_h$ (h) & 29571.59 \\ \hline
    \end{tabular}
    \caption{Bảng thông số bộ truyền đai}
\end{table}
\cleardoublepage