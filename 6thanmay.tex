\chapter{Chọn thân máy và các chi tiết phụ}
\section{Chọn thân máy}
\subsection{Yêu cầu}
\begin{itemize}
    \item Chỉ tiêu cơ bản của hộp giảm tốc là khối lượng nhỏ và độ cứng cao.
    \item Vật liệu làm vỏ là gang xám GX15-32.
    \item Hộp giảm tốc bao gồm: thành hộp, nẹp hoặc gân, mặt bích, gối đỡ, \ldots
    \item Bề mặt lắp ghép giữa nắp và thân được cạo sạch hoặc mài để lắp sít, khi lắp có một lớp sơn mỏng hoặc sơn đặc biệt.
    \item Chọn bề mặt ghép nắp và thân: song song mặt đế.
    \item Mặt đáy về phía lỗ tháo dầu với độ dốc khoảng $2^\circ$ và ngay tại chỗ tháo dầu lõm xuống.
\end{itemize}

\subsection{Xác định kích thước vỏ hộp}

\begin{table}[H]
\centering
\renewcommand{\arraystretch}{1.4}
\begin{tabular}{|l|l|l|}
\hline
\textbf{Tên gọi} & \textbf{Tên gọi} & \textbf{Kết quả} \\ \hline
\multirow{3}{*}{Chiều dày} 
    & Thân hộp, $\delta$ & $\delta = 0.03a + 3 = 9\ \text{mm}$ \\ \cline{2-3}
    & Nắp hộp, $\delta_1$ & $\delta_1 = 0.9\delta = 8\ \text{mm}$ \\ \cline{2-3}
    & Chiều dày, $e$ & $e = (0.8 \div 1)\delta = 8\ \text{mm}$ \\ \hline
\multirow{2}{*}{Gân tăng cứng} 
    & Chiều cao, $h$ & Chọn $h = 58\ \text{mm} \ (h < 58)$ \\ \cline{2-3}
    & Độ dốc & $2^\circ$ \\ \hline
\multirow{5}{*}{Đường kính} 
    & Bu lông nền, $d_1$ & $d_1 > 0.04a + 10 \Rightarrow d_1 = 16\ \text{mm}$ \\ \cline{2-3}
    & Bu lông cạnh ổ, $d_2$ & $d_2 = (0.7 \div 0.8)d_1 = 12\ \text{mm}$ \\ \cline{2-3}
    & Bu lông ghép bích nắp và thân, $d_3$ & $d_3 = (0.8 \div 0.9)d_2 = 10\ \text{mm}$ \\ \cline{2-3}
    & Vít ghép nắp ổ, $d_4$ & $d_4 = (0.6 \div 0.7)d_2 = 10\ \text{mm}$ \\ \cline{2-3}
    & Vít ghép nắp cửa thăm, $d_5$ & $d_5 = (0.5 \div 0.6)d_2 = 8\ \text{mm}$ \\ \hline
\multirow{4}{*}{Chiều rộng mặt bích} 
    & $K_1 = e_1 + (1.3 \div 1.4)d_2$ & $= 27\ \text{mm}$ \\ \cline{2-3}
    & $K_2 = (1.3 \div 1.4)d_2$ & $= 24\ \text{mm}$ \\ \cline{2-3}
    & $K_3 = e_3 + (1.3 \div 1.4)d_3$ & $= 25\ \text{mm}$ \\ \cline{2-3}
    & $K_4 = (1.3 \div 1.4)d_3$ & $= 25\ \text{mm}$ \\ \hline
    \multirow{4}{*}{Kích thước gối trục} 
    & Bề rộng mặt ghép bu lông cạnh ổ, $K_2$ & $K_2 = E_2 + R_2 + (3 \div 5) = 45\ \text{mm}$ \\ \cline{2-3}
    & $E_2 \approx 1.6d_2$ & $= 22\ \text{mm}$ \\ \cline{2-3}
    & $R_2 \approx 1.3d_2$ & $= 18\ \text{mm}$ \\ \cline{2-3}
    & $C_1 \approx D_{31}/2 = 66\ \text{mm}$, $C_2 \approx D_{32}/2 = 85\ \text{mm}$ & \\ \hline
Mặt đế hộp & Chiều cao $h$ & Theo kết cấu và kích thước mặt tựa \\ \hline
\multirow{4}{*}{Mặt đế hộp} 
    & $S_1 = (1.3 \div 1.5)d_1$ & $= 26\ \text{mm}$ \\ \cline{2-3}
    & Có phần lõi: $D_d$ theo dao khoét & $S_1 \approx (1.4 \div 1.7)d_1 = 28\ \text{mm}$ \\ \cline{2-3}
    & $S_2 \approx (1 \div 1.1)d_1 = 20\ \text{mm}$ & \\ \cline{2-3}
    & $K_1 \approx 3d_1 = 60\ \text{mm}$, $q \geq K_1 + 2\delta = 78\ \text{mm}$ & \\ \hline
\multirow{4}{*}{Khe hở giữa các chi tiết} 
    & Giữa bánh răng và thành trong & $\Delta \geq (1 \div 1.2)\delta = 64\ \text{mm}$ \\ \cline{2-3}
    & Giữa đỉnh răng lăn với đáy hộp & $\Delta_1 = (3 \div 5)\delta = 44\ \text{mm}$ \\ \cline{2-3}
    & Mặt bên bánh răng trục 1 & $\Delta \geq \delta \Rightarrow \Delta = 13\ \text{mm}$ \\ \cline{2-3}
    & Mặt bên bánh răng trục 2 & $\Delta \geq \delta \Rightarrow \Delta = 10\ \text{mm}$ \\ \hline
Số lượng bu lông nền & $Z = \frac{L + B}{200 \div 300} = 4$ & Với $L = 497\ \text{mm}$, $B = 233\ \text{mm}$ \\ \hline
\end{tabular}
\end{table}

\textbf{Kích thước gối trục: }
Tra bảng 18.2 tài liệu [2], ta có đường kính ngoài và tâm lỗ vít như sau:

\begin{table}[H]
\centering
\begin{tabular}{|c|c|c|c|}
\hline
\textbf{Trục} & \textbf{D} & \textbf{D2} & \textbf{D3} \\ \hline
I   & 72  & 102 & 130 \\ \hline
II  & 110 & 140 & 170 \\ \hline
\end{tabular}
\end{table}

\section{Các chi tiết phụ}
\subsection{Nắp ổ}
Nắp ổ thường chế tạo bằng gang xám GX15-32, được chọn như sau:

\begin{table}[H]
\centering
\begin{tabular}{|c|c|c|c|c|c|c|}
\hline
\textbf{Trục} & \textbf{d} & \textbf{D2} & \textbf{D3} & \textbf{a} & \textbf{b} & \textbf{c} \\ \hline
I   & 30 & 43  & 31   & 4.3 & 6 & 12 \\ \hline
II  & 50 & 69  & 51.5 & 6.5 & 9 & 16 \\ \hline
\end{tabular}
\end{table}

\subsection{Cửa thăm}
Để kiểm tra, quan sát các chi tiết trong hộp khi lắp ghép và đổ dầu vào hộp trên đỉnh hộp có làm cửa thăm. Cửa thăm được đậy bằng nắp. Trên nắp có thể lắp thêm nút thông hơi. Kích thước cửa thăm chọn theo bảng sau:

\begin{table}[H]
\centering
\begin{tabular}{|c|c|c|c|c|c|c|c|c|c|}
\hline
\textbf{A} & \textbf{B} & \textbf{C} & \textbf{D} & \textbf{E} & \textbf{F} & \textbf{G} & \textbf{R} & \textbf{Kích thước vít} & \textbf{Số lượng vít} \\ \hline
100 & 75 & 150 & 100 & 125 & -- & 87 & 12 & M8 & 4 \\ \hline
\end{tabular}
\end{table}

\subsection{Nút thông hơi}
Khi làm việc nhiệt độ trong hộp tăng lên, để giảm áp suất và điều hòa không khí bên trong và bên ngoài hộp, người ta dùng nút thông hơi. Nút thông hơi được lắp trên nắp cửa thăm hoặc ở vị trí cao nhất của nắp hộp. Ta chọn kích thước nút thông hơi có lưới chống bụi theo bảng sau:

\begin{table}[H]
\centering
\begin{tabular}{|c|c|c|c|c|c|c|c|c|c|c|c|c|c|c|c|}
\hline
\textbf{A} & \textbf{B} & \textbf{C} & \textbf{D} & \textbf{E} & \textbf{F} & \textbf{G} & \textbf{H} & \textbf{I} & \textbf{J} & \textbf{K} & \textbf{L} & \textbf{M} & \textbf{N} & \textbf{O} \\ \hline
M27X2 & 15 & 30 & 15 & 36 & 32 & 6 & 4 & 18 & 8 & 6 & 22 & 36 & 32 & 10 \\ \hline
\end{tabular}
\end{table}
Để vận chuyển hộp giảm tốc được thuận lợi, nên sử dụng bu lông vòng lắp trên
nắp hộp giảm tốc. Số lượng và kích thước bu lông vòng chọn theo trọng lượng hộp
giảm tốc và cách mắc dây cáp vào bu lông vòng

\subsection{Que thăm dầu}
Để kiểm tra mức dầu trong hộp, nên sử dụng que thăm dầu. Nên kiểm tra mức dầu khi hộp giảm tốc không hoạt động. Nếu hộp giảm tốc làm việc liên tục (3 ca/ngày) thì nên kèm theo ống bao bên ngoài để có thể kiểm tra mức dầu khi hộp giảm tốc đang hoạt động. Que thăm dầu nên đặt nghiêng so với phương thẳng đứng góc nhỏ hơn $35^\circ$. Kích thước que thăm dầu có dạng như sau:

\subsection{Nút tháo dầu}
Sau một thời gian làm việc, dầu bôi trơn chứa trong hộp bị bẩn hoặc bị biến chất,
do đó phải thay dầu mới. Để tháo dầu cũ, ở đáy hộp có lỗ tháo dầu. Lúc làm việc lỗ
được bít kín bằng nút tháo dầu. Ta chọn nút tháo dầu ren thẳng như sau:
\begin{table}[H]
    \centering
    \begin{tabular}{|c|c|c|c|c|c|c|c|c|c|c|}
    \hline
    $d$ & $b$ & $m$ & $f$ & $l$ & $c$ & $q$ & $D$ & $S$ & $D_0$ \\
    \hline
    M20 × 2 & 15 & 9 & 3 & 28 & 2.5 & 17.8 & 30 & 22 & 25.4 \\
    \hline
    \end{tabular}
    \end{table}

\subsection{Vòng chắn dầu}
Vòng gồm 3 rãnh tiết diện tam giác có góc ở đỉnh là $60^\circ$. Khoảng cách giữa các
đỉnh là 3 (mm). Vòng cách mép trong thành hộp khoảng (0,5÷1) mm. Khe hở giữa vỏ
với mặt ngoài của vòng ren là 0,43 (mm)

\subsection{Chốt định vị trụ}
Mặt ghép giữa nắp và thân nằm trong mặt phẳng chứa đường tâm các trục.
Lỗ trụ lắp ở trên nắp và thân hộp được gia công đồng thời, để đảm bảo vị trí
tương đối của nắp và thân trước và sau khi gia công cũng như khi lắp ghép,
ta dùng 2 chốt định vị, nhờ có chốt định vị khi xiết bulông không làm biến
dạng vòng ngoài của ổ. Ta chọn chốt định vị theo bảng sau: \\
\begin{center}
\begin{tabular}{|c|c|c|}
    \hline
    d & c & l \\
    \hline
    6 & 6 & 10 \\
    \hline
\end{tabular}
\end{center}

\subsection{Ống lót}
Ống lót được dùng để đỡ ổ lăn, để thuận tiện khi lắp ghép và điều chỉnh bộ phận ổ, đồng thời tránh cho ổ khỏi bụi bặm, chất bẩn. Ta chọn kích thước của ống lót có chiều dày:
$\delta = 6 \div 8 \text{ mm, ta chọn } \delta = 8 \text{ mm}$

\subsection{Vòng phớt}
Vòng phớt là loại lót kín động gián tiếp nhằm mục đích bảo vệ ổ khỏi bụi bặm, chất
bẩn, hạt cứng và các tạp chất khác xâm nhập vào ổ. Những chất này làm ổ chóng bị mài
mòn và bị han gỉ. Ngoài ra, vòng phớt còn đề phòng dầu chảy ra ngoài. Tuổi thọ ổ lăn phụ
thuộc rất nhiều vào vòng phớt.
Vòng phớt được dùng khá rộng rãi do có kết cấu đơn giản, thay thế dễ dàng. Tuy nhiên
có nhược điểm là chóng mòn và ma sát lớn khi bề mặt trục có độ nhám cao.

