\section{Chọn bộ truyền đai}
\subsection{Chọn loại đai}
Dựa vào công suất động cơ là $P_{dc} = 4.275 kW$ và số vòng quay $n_{dc} = 1450$ vòng/phút \\
$\Rightarrow$ Chọn đai loại A \\    

\begin{tabular}{|c|c|c|c|c|c|c|c|}
    \hline 
    Ký hiệu đai & $b_p$  & $b_0$  & h  & $y_0$ (mm) & A ($mm^2$) & Chiều dài đai (m) & $d_{1min} (mm)$ \\ \hline
    A & 11 & 13 & 8 & 2.8 & 81 & 560 $\div$ 4000 & 90 \\ \hline
\end{tabular}
\subsection{Tính đường kính bánh đai nhỏ}
Đường kính bánh đai nhỏ $d_1 = 1,2d_{min}$ với $d_{min} = 90 (mm)$. Vậy $d_1 = 90 \cdot 1.2=108(mm)$.
Chọn theo dãy giá trị tiêu chuẩn, ta chọn $d_1 = 112 (mm)$.\\
Vận tốc dài trên bánh đai nhỏ:\\
\[
    v_1 = \frac{\pi.d_1.n_1}{60000} = \frac{\pi.112.1450}{60000} = 8.503 (m/s) < 25 (m/s)
\]  
$\Rightarrow$ Thỏa điều kiện < 25 (m/s) \\
\subsection{Chọn hệ số trượt tương đối và tính đường kính bánh đai lớn}
Chọn hệ số trượt tương đối $\xi = 0.01$ \\
Từ công thức tỉ số của bộ truyền đai: \\
\[
    u_d = \frac{d_2}{d_1(1 - \xi)}
\]
\[
    \Rightarrow d_2 = u_d.d_1(1 - \xi) = 2.44 \cdot 112(1 - 0,01) = 270.55 (mm)
\]
Chọn theo dãy giá trị tiêu chuẩn, ta chọn $d_2 = 280 (mm)$ \\
Tính lại tỷ số truyền: \\
\[
    u_d = \frac{d_2}{d_1(1 - \xi)} = \frac{280}{112 \cdot 0,99} = 2.52
\]
Để sai số tỷ số truyền bằng 0, ta tính lại đường kính bánh đai nhỏ: \\
\[
    d_1 = \frac{d_2}{u_d \cdot (1- \xi)} = \frac{280}{2.44 \cdot (1-0.01)} = 115.91 (mm)
\]
\subsection{Chọn khoảng cách trục a}
Theo các thông số $u_d = 2.44$ và $d_2 = 280mm$ $\Rightarrow a = 1,2d_2 = 336 (mm)$ 
Chiều dài đai: \\
\[
    L = 2a + \pi\frac{(d_1 + d_2)}{2} + \frac{(d_2 - d_1)^2}{4a} = 2.336 + \pi\frac{(115.91 + 280)}{2} + \frac{(280 - 115.91)^2}{4.336} = 1313.93 (mm)
\]
$\Rightarrow$ Chọn chiều dài đai L = 1400 mm theo dãy giá trị tiêu chuẩn. \\
Tính lại khoảng cách trục: 
\[
    k = L - \pi\frac{d_1 + d_2}{2} = 1400 - \pi\frac{115.91 + 280}{2} =778.106 (mm)
\]
\[
    \Delta = \frac{d_2 - d_1}{2} = \frac{280 - 115.91}{2} = 82.045 (mm)
\]
\[
    a = \frac{k + \sqrt{k^2 - 8\Delta^2}}{4} = \frac{778.106 + \sqrt{778.106^2 - 8\cdot82.045^2}}{4} = 380.2 (mm)
\]
Kiểm tra a thỏa điều kiện nếu giá trị a vừa tính thỏa giá trị khoảng cách trục nhỏ
nhất được xác định theo công thức : \\
\[
    2(d_1+d_2) \geq a \geq 0,7(d_1+d_2)
\]
\[
    2(115.91+280) \geq a \geq 0,7(115.91+280)
\]
\[
    791.82 \geq a \geq 277.173
\]
$\Rightarrow$ a = 380.2 (mm) thỏa điều kiện. \\
\cleardoublepage
\subsection{Tính toán vận tốc đai và số truyền đai}
Vận tốc dây đai: \\
\[
    v = \frac{\pi.d_1.n_1}{60000} = \frac{\pi \cdot 115.91 \cdot 1450}{60000} = 8.8 (m/s)
\]
Số vòng chạy đai trong 1 giây: \\
\[
    i = \frac{v_1}{L} = \frac{8.8}{1400.10^{-3}} = 6.286s^{-1} 
\]
$\Rightarrow$ Thỏa điều kiện $i \leq [i] = 10s^{-1}$ \\
\subsection{Tính góc ôm đai bánh nhỏ}
\[
    \alpha_1 = 180 - 57\frac{d_2 - d_1}{a} = 180 - 57\frac{280 - 115.19}{380.2} = 155.29^{\circ}
\]
\subsection{Các hệ số sử dụng}
Hệ số xét đến ảnh hưởng góc ôm đai: \\
\[
    C_{\alpha} = 1.24(1 - e^{\frac{-\alpha_1}{110}}) = 1.24(1 - e^{\frac{-155.29}{110}}) = 0.938 \\
\]
Hệ số xét đến ảnh hưởng vận tốc: \\
\[
    C_v = 1 - 0,05(0,01v^2 - 1) = 1 - 0,05(0,01.8.8^2 - 1) = 1,011 \\
\]
Hệ số xét đến ảnh hưởng tỷ số truyền u: \\
\[
    C_u = 1,14 (v > 2.5 m/s)
\]
Hệ số xét đến ảnh hưởng của chiều dài đai L: \\
\[
    C_L = \sqrt[6]{\frac{L}{L_0}} = \sqrt[6]{\frac{1400}{1700}} = 0,968
\]
Hệ số xét đến sự ảnh hưởng của sự phân bố không đều tải trọng giữa các dây đai: \\
Chọn sơ bộ $C_z = 0.95$ \\
Hệ số xét đến ảnh hưởng của chế độ tải trọng: \\ 
Chọn sơ bộ $C_r = 0.9$ \\   
Chọn công suất có ích cho phép theo GOST 1284.3 - 96, ta có: \\
Đai loại A, $d_1 = 115.91 mm$, $v_1 = 8.8 m/s$ 
$\Rightarrow$ Chọn $[P_0] = 1,80$ \\
Tính số dây đai theo công thức: \\
\[
    z \geq \frac{P_1}{[P_0].C_{\alpha}.C_u.C_L.C_z.C_r.C_v} = \frac{4.102}{1.8\cdot 0.938\cdot 1.14\cdot 0.968\cdot 0.95\cdot 0.9\cdot 1,011} = 2.55
\]
$\Rightarrow$ Chọn z = 3 dây đai\\
Kiểm nghiệm lại $C_z$: vì z = 3 nên $C_z$ = 0,95 như đã chọn sơ bộ.
\subsection{Lực trên dây đai}
Tổng lực căn đai ban đầu trên cả  dây đai: \\
\[
    F_0 = z\cdot A\cdot [\sigma_0] = 3\cdot 81\cdot 1 = 243 (N)
\]
Trong đó: Đối với đai thang, $\sigma_0 \leq 1,5$ MPa nên ta chọn $\sigma_0 = 1$ MPa, z = 3, $A_0$ = 81 $mm^2$ \\
Lực căng trên mỗi dây đai: \\
\[
    \frac{F_0}{z} = \frac{243}{3} = 81 (N)
\]
Tổng lực vòng có ích trên cả 3 đai: \\
\[
    F_t =\frac{1000P_1}{v_1} = \frac{1000\cdot 4.102}{8.8} = 466.136 (N)
\]
Lực vòng có ích trên mỗi dây đai: \\
\[
    \frac{F_t}{z} = \frac{466.136}{3} = 155.379 (N)
\]
\subsection{Lực tác dụng lên trục}
\[
    F_r \approx 2F_0\sin(\frac{\alpha_1}{2}) = 2\cdot 243\sin(\frac{155.29}{2}) = 474.744 (N) 
\]
\subsection{Ứng suất lớn nhất trong dây đai}
\[
    \sigma_{max} = \sigma_1 + \sigma_v + \sigma_{F1} = \sigma_0 + 0,5\sigma_t + \sigma_v + \sigma_{F1} 
\]
\[
    = \frac{F_0}{A} + 0.5\cdot \frac{F_t}{A} + \rho \cdot v^2\cdot 10^{-6} + E\cdot \frac{2\cdot y_0}{d_1} 
\]
\[
    = \frac{243}{3\cdot 81} + 0.5\cdot \frac{466.136}{3\cdot 81} + 1000\cdot 8.8^2\cdot 10^{-6} + 60\cdot \frac{2\cdot 2.8}{115.19} = 4.88 (MPa) \\
\]
\subsection{Tuổi thọ dây đai}
\[
    L_h = \frac{(\frac{\sigma_r}{\sigma_{max}})^m\cdot 10^7}{2\cdot 3600\cdot i} = \frac{(\frac{9}{6.9})^8\cdot10^7}{2\cdot3600\cdot6.286} = 29571.59 (h)
\]
Trong đó:\\
$\sigma_r$ = 9 (MPa) - giới hạn mỏi của đai thang.\\
m = 8 - chỉ số mũ của đường cong mỏi đối với đai thang\\
i = 6.286 ($s^{-1}$) - số vòng chạy của đai trong một giây\\
\cleardoublepage