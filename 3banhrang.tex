\section{Tính toán hộp giảm tốc bánh răng nghiêng 1 cấp}
\subsection{Chọn vật liệu cho bánh dẫn và bánh bị dẫn}
Dựa vào bảng 6.1 [1], chọn vật liệu chế tạo cặp bánh răng là thép C45 tôi cải thiện với độ cứng bề mặt là $HB_1=260$ cho bánh dẫn và $HB_2=240$ cho bánh bị dẫn.

\subsection{Giới hạn mỏi tiếp xúc và giới hạn mỏi uốn}
\[
    \sigma_{OH_{lim}} = 2HB + 70 
\]
\[
    \Rightarrow \sigma_{OH_{lim1}} = 2.260 + 70 = 590MPa
\]
\[
    \Rightarrow \sigma_{OH_{lim2}} = 2.240 + 70 = 550MPa
\]
\[
    \sigma_{OF_{lim}} = 1.8HB
\]
\[
    \Rightarrow \sigma_{OF_{lim1}} = 1.8.260 = 468MPa
\]
\[
    \Rightarrow \sigma_{OF_{lim2}} = 1.8.240 = 432MPa
\]

\subsection{Tính toán các hệ số tuổi thọ và hệ số an toàn}
\subsubsection{Số chu kỳ thay đổi ứng suất tiếp xúc cơ sở}
\begin{center}
        $N_{HO_1} = 30.HB_1^{2,4} = 30.260^{2,4} = 1,875.10^7$ chu kỳ \\
        $N_{HO_2} = 30.HB_2^{2,4} = 30.240^{2,4} = 1,547.10^7$ chu kỳ \\
\end{center}
\subsubsection{Số chu kỳ thay đổi ứng suất uốn cơ sở}
\[
    N_{FO_1} = N_{FO_2} = 4 \cdot 10^6
\]

\subsubsection{Số chu kỳ thay đổi ứng suất tương đương}
Do bộ truyền chịu tải trọng tĩnh nên: 
\[
    N_{FE_1} = N_{HE_1} = 60\cdot c \cdot n_1 \cdot L_h = 1198.03\cdot 10^6
\]
\[
    N_{FE_2} = N_{HE_2} = 60\cdot c \cdot n_2 \cdot L_h = 2396.06\cdot 10^6
\]
Do $N_{HE_1} > N_{HO_1}, N_{HE_2} > N_{HO_2}$, ta có thể xem như $K_{HL} = 1$ \\
Tương tự ta có $K_{FL} = 1$ \\
Do bộ truyền quay 1 chiều nên $K_{FC} = 1$ \\
Với độ cứng bề mặt của vật liệu làm bánh răng, từ bảng 6.2 [1], chọn $s_F=1.75, s_H=1.1$ \\
\subsection{Ứng suất cho phép}
\subsubsection{Ứng suất tiếp xúc cho phép của bánh dẫn và bánh bị dẫn}
\[
    [\sigma_{H_1}] = \frac{K_{HL_1}\sigma_{OH_{lim1}}}{s_H} = 536.36MPa
\]
\[
    [\sigma_{H_2}] = \frac{K_{HL_2}\sigma_{OH_{lim2}}}{s_H} = 500MPa
\]
\subsubsection{Ứng suất uốn cho phép của bánh dẫn và bánh bị dẫn}
\[
    [\sigma_{F_1}] = \frac{K_{FL_1}\sigma_{OF_{lim1}}K_{FC}}{s_F} = 267.43MPa
\]
\[
    [\sigma_{F_2}] = \frac{K_{FL_2}\sigma_{OF_{lim2}}K_{FC}}{s_F} = 246.86MPa
\]
Như vậy, ứng suất tiếp xúc cho phép của cả bộ truyền là:
\[
    [\sigma_H] = \frac{[\sigma_{H_1}] + [\sigma_{H_2}]}{2} = 518.18MPa 
\]
Ứng suất tiếp xúc cho phép khi quá tải:
\[
    [\sigma_H]_{max} = 2.8 \cdot \sigma_{ch} = 2.8 \cdot 650 = 1820MPa
\]
Ứng suất uốn cho phép khi quá tải:
\[
    [\sigma_F]_{max} = 0.8 \cdot \sigma_{ch} = 0.8 \cdot 650 = 520MPa
\]
% \subsection{Khoảng cách trục}
% Do bánh răng nằm đối xứng các ổ trục nên $\Psi_{ba} = 0,3 \div 0,5$, chọn $\Psi_{ba} = 0,4$ \\
% Khi đó $\Psi_{bd} = \frac{\Psi_{ba} \cdot (u+1)}{2} = \frac{0,4 \cdot (6,3+1)}{2} = 1,46$ \\
% Từ đó tra theo bảng 6.4 ta chọn: $K_{H\beta} = 1,07$, $K_{F\beta} = 1,14$ \\
% Tính khoảng cách trục: \\
% \[
%     a_w = 500 \cdot (u+1) \cdot \sqrt[3]{\frac{T_1 \cdot K_{H\beta}}{\Psi_{ba} \cdot [\sigma_H]^2 \cdot u}} = 500 \cdot (6,3+1) \cdot \sqrt[3]{\frac{63,73 \cdot 1,07}{0,4 \cdot 408,68^2 \cdot 6,3}} = 198,98 (mm)
% \]
% Theo tiêu chuẩn ta chọn $a_w = 200mm$
% \subsection{Môđun răng}
% \[
%     m_n = (0,01 \div 0,02) \cdot a_w = 2 \div 4
% \]
% Theo tiêu chuẩn ta chọn $m_n = 4$
% \subsection{Số răng}
% Từ điều kiện $20^\circ \leq \beta \leq 8^\circ$ \\
% \[
%     \Leftrightarrow \frac{2 \cdot a_w \cdot \cos8^\circ}{m_n \cdot (u+1)} \leq z_1 \leq \frac{2 \cdot a_w \cdot \cos20^\circ}{m_n \cdot (u+1)}
% \]
% \[
%     \Leftrightarrow \frac{2 \cdot 200 \cdot \cos8^\circ}{4 \cdot (6,3+1)} \leq z_1 \leq \frac{2 \cdot 200 \cdot \cos20^\circ}{4 \cdot (6,3+1)}
% \]
% \[
%     \Leftrightarrow 13,56 \leq z_1 \leq 12,87
% \]
% $\Rightarrow$ Chọn $z_1 = 13 \Rightarrow z_2 = u.z_1 = 6,3.13 = 81,9$ \\
% $\Rightarrow$ Chọn $z_2 = 82$ \\
% Góc nghiêng răng: 
% \[
%     \beta = \arccos\frac{m_n \cdot (z_1+z_2)}{2 \cdot a_w} = \arccos\frac{4 \cdot (13+82)}{2 \cdot 200} = 18,2^\circ
% \]
% \subsection{Tỉ số truyền sau khi chọn răng}
% \[
%     u = \frac{z_2}{z_1} = \frac{82}{13} = 6,308
% \]
% \subsection{Các thông số hình học}
% Đường kính vòng chia: \\
% \[
%     d_1 = \frac{m_n \cdot z_1}{\cos\beta} = \frac{4 \cdot 13}{\cos18,2^\circ} = 54,738 (mm)
% \]
% \[
%     d_2 = \frac{m_n \cdot z_2}{\cos\beta} = \frac{4 \cdot 82}{\cos18,2^\circ} = 345,27 (mm)
% \]
% Đường kính vòng đỉnh: \\
% \[
%     d_{a1} = d_1 + 2 \cdot m_n = 54,738 + 2 \cdot 4 = 62,738 (mm)
% \]
% \[
%     d_{a2} = d_2 + 2 \cdot m_n = 354,27 + 2 \cdot 4 = 362,27 (mm)
% \]
% Đường kính vòng đáy: \\ 
% \[
%     d_{f1} = d_1 - 2,5 \cdot m_n = 62,738 - 2,5 \cdot 4 = 44,738 (mm)
% \]
% \[
%     d_{f2} = d_2 - 2,5 \cdot m_n = 345,27 - 2,5 \cdot 4 = 335,27 (mm)
% \]
% Tính lại khoảng cách trục: \\
% \[
%     a_w = \frac{m_n \cdot (z_1+z_2)}{2 \cdot \cos\beta} = \frac{4 \cdot (13+82)}{2 \cdot \cos18,2^\circ} = 200(mm)
% \]
% Chiều rộng vành răng: 
% \begin{itemize}
%     \item Bánh bị dẫn: $b_2 = \Psi_{ba} \cdot a_w = 0,4 \cdot 200 = 80(mm)$
%     \item Bánh dẫn: $b_1 = b_2 + 5 = 85(mm)$
% \end{itemize}
% \subsection{Vận tốc vòng bánh răng}
% \[
%     v = \frac{\pi \cdot d_1 \cdot n_1}{60000} = \frac{\pi \cdot 54,738 \cdot 482,5}{60000} = 1,38 (m/s)
% \]
% Theo bảng 6.3 ta chọn cấp chính xác 9 với $v_{gh} = 6 m/s$
% \subsection{Tính toán kiểm nghiệm giá trị ứng suất tiếp xúc}
% Theo bảng 6.6 ta chọn hệ số tải trọng động $K_{HV} = 1,11$, $K_{FV} = 1,22$ \\
% \[
%     \sigma_H = \frac{z_M \cdot z_H \cdot z_\epsilon}{d_1} \cdot \sqrt{\frac{2 \cdot T_1 \cdot 10^3 \cdot K_{H\beta} \cdot K_{HV} \cdot (u+1)}{b_w \cdot u}} 
% \]
% \[
%     = \frac{190 \cdot 2,3847 \cdot 0,8124}{54,738} \cdot \sqrt{\frac{2 \cdot 63,73 \cdot 10^3 \cdot 1,07 \cdot 1,11 \cdot (6,3+1)}{80 \cdot 6,3}} = 314,887 (MPa)
% \]
% Với:
% \begin{itemize}
%     \item $Z_M = 190$ vì vật liệu bằng thép
%     \item $Z_H = \sqrt{\frac{4 \cdot \cos\beta}{\sin(2 \cdot \alpha_{tw})}} = \sqrt{\frac{4 \cdot \cos18,2}{\sin(2 \cdot 20,9637)}} = 2,3847$
%     \item $\alpha_{tw} = \arctan(\frac{\tan\alpha_{nw}}{\cos\beta}) = \arctan(\frac{\tan20}{\cos18,2}) = 20,9637$
%     \item $Z_\epsilon = \sqrt{\frac{1}{\epsilon_\alpha}} = \sqrt{\frac{1}{1,515}} = 0,8124$
%     \item $\epsilon_\alpha = [1,88-3,2 \cdot (\frac{1}{z_1}+\frac{1}{z_2})] \cdot \cos\beta = [1,88-3,2 \cdot (\frac{1}{13}+\frac{1}{82})] \cdot \cos18,2 = 1,515$
% \end{itemize}
% \[
%     \sigma_H = 314,887 MPa \leq 441,82 MPa
% \]
%     $\Rightarrow \sigma_H \leq [\sigma_H]$ thỏa mãn\\
% \subsection{Số răng tương đương}
% \[
%     z_{v1} = \frac{z_1}{\cos^3(\beta)} = \frac{13}{\cos^3(18,2)} = 13,685 \Rightarrow z_{v1} = 14
% \]
% \[
%     z_{v2} = \frac{z_2}{\cos^3(\beta)} = \frac{82}{\cos^3(18,2)} = 86,32 \Rightarrow z_{v1} = 87
% \]
% \subsection{Hệ số dạng răng}
% \begin{itemize}
%     \item Đối với bánh dẫn: $Y_{F1} = 3,47 + \frac{13,2}{z_{v1}} = 3,47 + \frac{13,2}{14} = 4,413$
%     \item Đối với bánh bị dẫn: $Y_{F2} = 3,47 + \frac{13,2}{z_{v2}} = 3,47 + \frac{13,2}{87} = 3,622$
% \end{itemize}
% \subsection{Ứng suất uốn tính toán}
% Đặc tính so sánh độ bền uốn:
% \begin{itemize}
%     \item Bánh dẫn: $\frac{[\sigma_{F1}]}{Y_{F1}} = \frac{257,143}{4,413} = 58,27$
%     \item Bánh bị dẫn: $\frac{[\sigma_{F2}]}{Y_{F2}} = \frac{241,7}{3,622} = 66,73$
% \end{itemize}
% $\Rightarrow$ Ta kiểm tra độ bền uốn theo bánh dẫn có độ bền thấp hơn:\\
% Ứng suất uốn tính toán:
% \[
%     \sigma_{F1} = \frac{Y_{F1} \cdot F_t \cdot K_{F1} \cdot Y_\epsilon \cdot Y_\beta}{b_2 \cdot m} = \frac{4,413 \cdot 2328,547 \cdot 1,39 \cdot 0,66 \cdot 0,6985}{80 \cdot 4} = 20,577 (MPa)
% \] 
% Với: 
% \begin{itemize}
%     \item $Y_\epsilon = \frac{1}{\epsilon_\alpha} = \frac{1}{1,515} = 0,66$
%     \item $F_t = 2 \cdot 10^3 \cdot \frac{T_1}{d_{w1}} = 2 \cdot 10^3 \cdot \frac{63,73}{54,738} = 2328,547$
%     \item $Y_\beta = 1 - \epsilon_\beta \cdot \frac{\beta}{120} = 1 - 1,988 \cdot \frac{18,2}{120} = 0,6985$
%     \item $\epsilon_\beta = b_w \cdot \frac{\sin\beta}{\pi \cdot m} = 80 \cdot \frac{\sin18,2}{\pi \cdot 4} = 1,988$
%     \item $K_{F1} = K_{F\alpha} \cdot K_{F\beta} \cdot K_{FV} = 1 \cdot 1,14 \cdot 1,22 = 1,39$
%     \item $K_{F\alpha} = \frac{4 + (\epsilon_\alpha - 1) \cdot (n_{cx} - 5)}{4 \cdot \epsilon_\alpha} = \frac{4 + (1,515 - 1) \cdot (9 - 5)}{4 \cdot 1,515} = 1$
% \end{itemize}
% $\Rightarrow \sigma_{F1} < [\sigma_{F1}] = 257,143$ MPa do đó độ bền uốn thỏa\\

% \cleardoublepage