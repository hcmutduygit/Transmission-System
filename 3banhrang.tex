\chapter{Tính toán hộp giảm tốc bánh răng nghiêng 1 cấp}
\section{Chọn vật liệu cho bánh dẫn và bánh bị dẫn}
Dựa vào bảng 6.1 [1], chọn vật liệu chế tạo cặp bánh răng là thép C45 tôi cải thiện với độ cứng bề mặt là $HB_1=260$ cho bánh dẫn và $HB_2=240$ cho bánh bị dẫn.

\section{Giới hạn mỏi tiếp xúc và giới hạn mỏi uốn}
\begin{equation}
    \sigma_{OH_{lim}} = 2HB + 70 
\end{equation}
\[
    \Rightarrow \sigma_{OH_{lim1}} = 2.260 + 70 = 590MPa
\]
\[
    \Rightarrow \sigma_{OH_{lim2}} = 2.240 + 70 = 550MPa
\]
\begin{equation}
    \sigma_{OF_{lim}} = 1.8HB
\end{equation}
\[
    \Rightarrow \sigma_{OF_{lim1}} = 1.8.260 = 468MPa
\]
\[
    \Rightarrow \sigma_{OF_{lim2}} = 1.8.240 = 432MPa
\]

\section{Tính toán các hệ số tuổi thọ và hệ số an toàn}
\subsection{Số chu kỳ thay đổi ứng suất tiếp xúc cơ sở}
\begin{equation}
        N_{HO_1} = 30.HB_1^{2,4} = 30.260^{2,4} = 1,875.10^7 \text{chu kỳ} \\
\end{equation}
\begin{equation}
        N_{HO_2} = 30.HB_2^{2,4} = 30.240^{2,4} = 1,547.10^7 \text{chu kỳ} \\
\end{equation}
\subsection{Số chu kỳ thay đổi ứng suất uốn cơ sở}
\begin{equation}
    N_{FO_1} = N_{FO_2} = 4 \cdot 10^6
\end{equation}

\subsection{Số chu kỳ thay đổi ứng suất tương đương}
Do bộ truyền chịu tải trọng tĩnh nên: 
\begin{equation}
    N_{FE_1} = N_{HE_1} = 60\cdot c \cdot n_1 \cdot L_h = 1198.03\cdot 10^6
\end{equation}
\begin{equation}
    N_{FE_2} = N_{HE_2} = 60\cdot c \cdot n_2 \cdot L_h = 2396.06\cdot 10^6
\end{equation}
Do $N_{HE_1} > N_{HO_1}, N_{HE_2} > N_{HO_2}$, ta có thể xem như $K_{HL} = 1$ \\
Tương tự ta có $K_{FL} = 1$ \\
Do bộ truyền quay 1 chiều nên $K_{FC} = 1$ \\
Với độ cứng bề mặt của vật liệu làm bánh răng, từ bảng 6.2 [1], chọn $s_F=1.75, s_H=1.1$ \\
\section{Ứng suất cho phép}
\subsection{Ứng suất tiếp xúc cho phép của bánh dẫn và bánh bị dẫn}
\begin{equation}
    [\sigma_{H_1}] = \frac{K_{HL_1}\sigma_{OH_{lim1}}}{s_H} = 536.36MPa
\end{equation}
\begin{equation}
    [\sigma_{H_2}] = \frac{K_{HL_2}\sigma_{OH_{lim2}}}{s_H} = 500MPa
\end{equation}
\subsection{Ứng suất uốn cho phép của bánh dẫn và bánh bị dẫn}
\begin{equation}
    [\sigma_{F_1}] = \frac{K_{FL_1}\sigma_{OF_{lim1}}K_{FC}}{s_F} = 267.43MPa
\end{equation}
\begin{equation}
    [\sigma_{F_2}] = \frac{K_{FL_2}\sigma_{OF_{lim2}}K_{FC}}{s_F} = 246.86MPa
\end{equation}
Như vậy, ứng suất tiếp xúc cho phép của cả bộ truyền là:
\begin{equation}
    [\sigma_H] = \frac{[\sigma_{H_1}] + [\sigma_{H_2}]}{2} = 518.18MPa 
\end{equation}
Ứng suất tiếp xúc cho phép khi quá tải:
\begin{equation}
    [\sigma_H]_{max} = 2.8 \cdot \sigma_{ch} = 2.8 \cdot 650 = 1820MPa
\end{equation}
Ứng suất uốn cho phép khi quá tải:
\begin{equation}
    [\sigma_F]_{max} = 0.8 \cdot \sigma_{ch} = 0.8 \cdot 650 = 520MPa
\end{equation}

\section{Khoảng cách trục}
Do bánh răng nằm đối xứng các ổ trục nên $\Psi_{ba} = 0.3 \div 0.5$, chọn $\Psi_{ba} = 0.4$  
Khi đó $\Psi_{bd} = 0.53\Psi_{ba}(u+1) = 1.272$  
Từ đó tra theo bảng 6.7 \cite{reference} ta chọn: $K_{HB} = 1.06$, $K_{FB} = 1.14$  \\
Khoảng cách trục sơ bộ:
\begin{equation}
a_w = 43 \cdot (u+1) \cdot \sqrt{\frac{T_1 \cdot K_{H\beta}}{\Psi_{ba} \cdot [\sigma_H]^2 \cdot u}} = 130.73(mm)
\end{equation}
Theo tiêu chuẩn ta chọn $a_w = 160mm$

\section{Môđun răng}
\begin{equation}
    m_n = (0.01 \div 0.02) \cdot a_w = 1.6 \div 3.2
\end{equation}
Theo tiêu chuẩn ta chọn $m_n = 3$
\section{Số răng}
Từ điều kiện $20^\circ \leq \beta \leq 8^\circ$
\[
\Leftrightarrow \frac{2 \cdot a_w \cdot \cos 8^\circ}{m_n \cdot (u+1)} \leq z_1 \leq \frac{2 \cdot a_w \cdot \cos 20^\circ}{m_n \cdot (u+1)}
\]
\[
\Leftrightarrow \frac{2 \cdot 160 \cdot \cos 8^\circ}{3 \cdot (5+1)} \leq z_1 \leq \frac{2 \cdot 160 \cdot \cos 20^\circ}{3 \cdot (5+1)}
\]
\[
\Leftrightarrow 16.7 \leq z_1 \leq 17.6
\]
\[
\Rightarrow \text{Chọn } z_1 = 17 \Rightarrow z_2 = u \cdot z_1 = 5.17 = 85
\]
\[
\Rightarrow \text{Chọn } z_2 = 85
\]
Góc nghiêng răng:
\begin{equation}
\beta = \arccos \left( \frac{m_n \cdot (z_1 + z_2)}{2 \cdot a_w} \right) = \arccos \left( \frac{3 \cdot (17+85)}{2 \cdot 160} \right) = 17.01^\circ
\end{equation}

\section{Các thông số hình học}

Đường kính lăn của bánh dẫn:
\begin{equation}
d_{w1} = \frac{2a_w}{m_n + 1} = \frac{2 \cdot 160}{5 + 1} = 53.33 mm
\end{equation}
Chiều rộng vành răng:
\begin{equation}
b_w = \Psi_{ba} \cdot a_w = 0.4 \cdot 160 = 64mm
\end{equation}
Tính lại khoảng cách trục:
\begin{equation}
a_w = \frac{m_n \cdot (z_1 + z_2)}{2 \cdot \cos \beta} = \frac{3 \cdot (17 + 85)}{2 \cdot \cos 17.01^\circ} = 160(mm)
\end{equation}
\[
\Rightarrow \text{Không cần dịch chỉnh răng, nên } d_{w1} = d_{w2} = d_2
\]
\section{Kiểm nghiệm bánh răng theo độ bền tiếp xúc}
\subsection{Xác định các hệ số tải trọng}
Vận tốc vòng của bánh dẫn:
\begin{equation}
v = \frac{\pi \cdot d_1 \cdot n_1}{60000} = \frac{\pi \cdot 53.33 \cdot 594.26}{60000} = 1.66(m/s)
\end{equation}
Theo bảng 6.13 [2], ta chọn cặp chính xác bậc 6, như vậy hệ số sai lệch bước răng $g_0 = 38$  \\
Theo bảng 6.15 [2], ta có $\delta_H = 0.002$, $\delta_F = 0.006$  \\
Theo bảng 6.14 [2], ta có hệ số phân bố không đều tải trọng giữa các đôi răng:\\
\begin{equation}
K_{H\alpha} = 1.04, \quad K_{F\alpha} = 1.13
\end{equation}
Vận tốc:
\begin{equation}
v_H = \delta_H \cdot g_0 \cdot v\sqrt{\frac{a_w}{u_{br}}} = 7.135(m/s)
\end{equation}
Hệ số tải trọng động trong vùng ăn khớp:
\begin{equation}
K_{Hv} = 1 + \frac{v_H \cdot b_w \cdot d_{w1}}{2T_1 \cdot K_{H\beta} \cdot K_{H\alpha}} = 1.181
\end{equation}
Như vậy hệ số tải trọng động tiếp xúc $K_H$:
\begin{equation}
K_H = K_{H\alpha} \cdot K_{H\beta} \cdot K_{Hv} = 1.302
\end{equation}

\subsection{Xác định các hệ số xét đến hình dạng, vật liệu và trùng khớp}
Với:
\begin{itemize}
    \item Hệ số kể đến của vật liệu $Z_M = 274 \text{ tra từ bảng 6.5 [2]}$
    \begin{equation}
    Z_H = \sqrt{\frac{2 \cos \beta_b}{\sin(2\cdot\alpha_{tw})}} = \sqrt{\frac{2\cdot cos15.96}{sin(2\cdot 20.84)}} = 1.7
    \end{equation}
    \item Góc profile răng
    \begin{equation}
    \alpha_{tw} = \arctan \left( \frac{tan\alpha_{nw}}{cos_\beta} \right) = \arctan \left( \frac{tan20}{cos17.01} \right) = 20.84^\circ
    \end{equation}
    \item Góc nghiêng răng trên hình trụ cơ sở
    \begin{equation}
    \beta_b = arctan(cos\alpha_t\cdot tan\beta) = 15.96^\circ
    \end{equation}
    \item Hệ số kể đến sự trùng khớp của răng
    \begin{equation}
    Z_\epsilon = \sqrt{\frac{1}{\epsilon_\alpha}} = \frac{1}{1.581} = 0.795
    \end{equation}
    \item Hệ số trùng khớp ngang 
    \begin{equation}
    \epsilon_\alpha = [1.88-3.2\cdot(\frac{1}{z_1}+\frac{1}{z_2})].cos\beta = [1.88 - 3.2\cdot (\frac{1}{17} + \frac{1}{85})]\cdot cos17.01 = 1.582
    \end{equation}
\end{itemize}
\section{Kiểm nghiệm bánh răng theo độ bền uốn}

\subsection{Hệ số tải trọng động trong vùng ăn khớp}

\begin{equation}
K_{Fv} = 1 + \frac{v_P \cdot b_w \cdot d_{w1}}{2 \cdot T_1 \cdot K_{FB} \cdot K_{Fa}} = 1.54
\end{equation}

Trong đó:

\begin{equation}
v_P = \delta_F \cdot g_0 \cdot \sqrt{\frac{v_{br}}{u}} = 21.4(m/s)
\end{equation}

Hệ số tải trọng uốn:

\begin{equation}
K_F = K_{Fa} \cdot K_{FB} \cdot K_{Fv} = 1.986
\end{equation}

\subsection{Hệ số trùng khớp răng}

\begin{equation}
Y_\epsilon = \frac{1}{\epsilon_\alpha} = 0.632
\end{equation}

\subsection{Hệ số kể đến độ nghiêng răng}

\begin{equation}
Y_\beta = 1 - \frac{\beta}{140} = 0.878
\end{equation}
Hệ số số răng $Y_{F1}, Y_{F2}$ tra theo số răng tương đương $z_{v1} = \frac{z_1}{\cos^3 \beta}$ từ bảng 6.18 \cite{reference}:

\begin{equation}
Y_{F1} = 4.08, \quad Y_{F2} = 3.6
\end{equation}

\subsection{Tính toán ứng suất uốn}

Như vậy ứng suất uốn của bánh dẫn và bánh bị dẫn:

\begin{equation}
\sigma_{F1} = \frac{2T_1 K_F Y_\epsilon Y_\beta Y_{F1}}{b_w d_{w1} m_n} = 53.75 MPa < [\sigma_{F1}]
\end{equation}
\begin{equation}
\sigma_{F2} = \frac{\sigma_{F1} Y_{F2}}{Y_{F1}} = 60.91 MPa < [\sigma_{F2}]
\end{equation}
Vậy thông số hình học của bộ truyền thỏa điều kiện bền uốn.
\subsection{Bảng thông số bộ truyền bánh răng}
\begin{table}[H]
    \centering
    \begin{tabular}{|l|c|c|c|}
        \hline
        \textbf{Thông số} & \textbf{Ký hiệu} & \textbf{Giá trị} & \textbf{Đơn vị} \\ \hline
        Tỷ số truyền & $u$ & 5 & - \\ \hline
        Khoảng cách trục & $a_w$ & 160 & mm \\ \hline
        Module & $m_n$ & 3 & mm \\ \hline
        Số răng bánh dẫn & $z_1$ & 17 & răng \\ \hline
        Số răng bánh bị dẫn & $z_2$ & 85 & răng \\ \hline
        Góc nghiêng răng & $\beta$ & 17.01 & $^\circ$ \\ \hline
        Đường kính vòng chia & $d_1, d_2$ & 53.33; 266.67 & mm \\ \hline
        Đường kính vòng đỉnh chia & $d_{a1}, d_{a2}$ & 59.33; 272.67 & mm \\ \hline
        Đường kính vòng chân & $d_{f1}, d_{f2}$ & 45.83; 259.17 & mm \\ \hline
        Bề dày bánh răng & $b_w$ & 70; 64 & mm \\ \hline
        Vận tốc vòng & $v$ & 1.66 & m/s \\ \hline
    \end{tabular}
    \caption{Bảng thông số hình học của bộ truyền bánh răng}
\end{table}
\subsection{Bôi trơn hộp giảm tốc}
Mức dầu thấp nhất:
\begin{equation}
    (0.75 \div 2)h = (0.75 \div 2) \cdot 6.5 = 5.06 \div 13.5 \, \text{mm}.
\end{equation}
Do mức dầu thấp nhất không thể nhỏ hơn $10 \, \text{mm}$, ta chọn mức dầu thấp nhất là $13.5 \, \text{mm}$. Phần bánh răng ngâm trong dầu phải nhỏ hơn:
\begin{equation}
    \frac{1}{3}d_{a2} = 90.89 \, \text{mm}.
\end{equation}
Chọn mức dầu cao nhất là $25 \, \text{mm}$, như vậy chiều cao phần bánh răng ngâm trong dầu là $25.53 \, \text{mm}$, thỏa mãn điều kiện trên.