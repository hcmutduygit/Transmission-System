\section{Tính toán hộp giảm tốc bánh răng nghiêng 1 cấp}
\subsection{Chọn vật liệu cho bánh dẫn và bánh bị dẫn}
Dựa vào bảng 6.1 [1], chọn vật liệu chế tạo cặp bánh răng là thép C45 tôi cải thiện với độ cứng bề mặt là $HB_1=260$ cho bánh dẫn và $HB_2=240$ cho bánh bị dẫn.

\subsection{Giới hạn mỏi tiếp xúc và giới hạn mỏi uốn}
\[
    \sigma_{OH_{lim}} = 2HB + 70 
\]
\[
    \Rightarrow \sigma_{OH_{lim1}} = 2.260 + 70 = 590MPa
\]
\[
    \Rightarrow \sigma_{OH_{lim2}} = 2.240 + 70 = 550MPa
\]
\[
    \sigma_{OF_{lim}} = 1.8HB
\]
\[
    \Rightarrow \sigma_{OF_{lim1}} = 1.8.260 = 468MPa
\]
\[
    \Rightarrow \sigma_{OF_{lim2}} = 1.8.240 = 432MPa
\]

\subsection{Tính toán các hệ số tuổi thọ và hệ số an toàn}
\subsubsection{Số chu kỳ thay đổi ứng suất tiếp xúc cơ sở}
\begin{center}
        $N_{HO_1} = 30.HB_1^{2,4} = 30.260^{2,4} = 1,875.10^7$ chu kỳ \\
        $N_{HO_2} = 30.HB_2^{2,4} = 30.240^{2,4} = 1,547.10^7$ chu kỳ \\
\end{center}
\subsubsection{Số chu kỳ thay đổi ứng suất uốn cơ sở}
\[
    N_{FO_1} = N_{FO_2} = 4 \cdot 10^6
\]

\subsubsection{Số chu kỳ thay đổi ứng suất tương đương}
Do bộ truyền chịu tải trọng tĩnh nên: 
\[
    N_{FE_1} = N_{HE_1} = 60\cdot c \cdot n_1 \cdot L_h = 1198.03\cdot 10^6
\]
\[
    N_{FE_2} = N_{HE_2} = 60\cdot c \cdot n_2 \cdot L_h = 2396.06\cdot 10^6
\]
Do $N_{HE_1} > N_{HO_1}, N_{HE_2} > N_{HO_2}$, ta có thể xem như $K_{HL} = 1$ \\
Tương tự ta có $K_{FL} = 1$ \\
Do bộ truyền quay 1 chiều nên $K_{FC} = 1$ \\
Với độ cứng bề mặt của vật liệu làm bánh răng, từ bảng 6.2 [1], chọn $s_F=1.75, s_H=1.1$ \\
\cleardoublepage
\subsection{Ứng suất cho phép}
\subsubsection{Ứng suất tiếp xúc cho phép của bánh dẫn và bánh bị dẫn}
\[
    [\sigma_{H_1}] = \frac{K_{HL_1}\sigma_{OH_{lim1}}}{s_H} = 536.36MPa
\]
\[
    [\sigma_{H_2}] = \frac{K_{HL_2}\sigma_{OH_{lim2}}}{s_H} = 500MPa
\]
\subsubsection{Ứng suất uốn cho phép của bánh dẫn và bánh bị dẫn}
\[
    [\sigma_{F_1}] = \frac{K_{FL_1}\sigma_{OF_{lim1}}K_{FC}}{s_F} = 267.43MPa
\]
\[
    [\sigma_{F_2}] = \frac{K_{FL_2}\sigma_{OF_{lim2}}K_{FC}}{s_F} = 246.86MPa
\]
Như vậy, ứng suất tiếp xúc cho phép của cả bộ truyền là:
\[
    [\sigma_H] = \frac{[\sigma_{H_1}] + [\sigma_{H_2}]}{2} = 518.18MPa 
\]
Ứng suất tiếp xúc cho phép khi quá tải:
\[
    [\sigma_H]_{max} = 2.8 \cdot \sigma_{ch} = 2.8 \cdot 650 = 1820MPa
\]
Ứng suất uốn cho phép khi quá tải:
\[
    [\sigma_F]_{max} = 0.8 \cdot \sigma_{ch} = 0.8 \cdot 650 = 520MPa
\]

\subsection{Khoảng cách trục}
Do bánh răng nằm đối xứng các ổ trục nên $\Psi_{ba} = 0.3 \div 0.5$, chọn $\Psi_{ba} = 0.4$  
Khi đó $\Psi_{bd} = 0.53\Psi_{ba}(u+1) = 1.272$  
Từ đó tra theo bảng 6.7 \cite{reference} ta chọn: $K_{HB} = 1.06$, $K_{FB} = 1.14$  \\
Khoảng cách trục sơ bộ:
\[
a_w = 43 \cdot (u+1) \cdot \sqrt{\frac{T_1 \cdot K_{H\beta}}{\Psi_{ba} \cdot [\sigma_H]^2 \cdot u}} = 130.73(mm)
\]
Theo tiêu chuẩn ta chọn $a_w = 160mm$

\subsection{Môđun răng}

\[
m_n = (0.01 \div 0.02) \cdot a_w = 1.6 \div 3.2
\]
Theo tiêu chuẩn ta chọn $m_n = 3$
\subsection{Số răng}
Từ điều kiện $20^\circ \leq \beta \leq 8^\circ$
\[
\Leftrightarrow \frac{2 \cdot a_w \cdot \cos 8^\circ}{m_n \cdot (u+1)} \leq z_1 \leq \frac{2 \cdot a_w \cdot \cos 20^\circ}{m_n \cdot (u+1)}
\]
\[
\Leftrightarrow \frac{2 \cdot 160 \cdot \cos 8^\circ}{3 \cdot (5+1)} \leq z_1 \leq \frac{2 \cdot 160 \cdot \cos 20^\circ}{3 \cdot (5+1)}
\]
\[
\Leftrightarrow 16.7 \leq z_1 \leq 17.6
\]
\[
\Rightarrow \text{Chọn } z_1 = 17 \Rightarrow z_2 = u \cdot z_1 = 5.17 = 85
\]
\[
\Rightarrow \text{Chọn } z_2 = 85
\]
Góc nghiêng răng:
\[
\beta = \arccos \left( \frac{m_n \cdot (z_1 + z_2)}{2 \cdot a_w} \right) = \arccos \left( \frac{3 \cdot (17+85)}{2 \cdot 160} \right) = 17.01^\circ
\]

\subsection{Các thông số hình học}

Đường kính lăn của bánh dẫn:
\[
d_{w1} = \frac{2a_w}{m_n + 1} = \frac{2 \cdot 160}{5 + 1} = 53.33 mm
\]
Chiều rộng vành răng:
\[
b_w = \Psi_{ba} \cdot a_w = 0.4 \cdot 160 = 64mm
\]
Tính lại khoảng cách trục:
\[
a_w = \frac{m_n \cdot (z_1 + z_2)}{2 \cdot \cos \beta} = \frac{3 \cdot (17 + 85)}{2 \cdot \cos 17.01^\circ} = 160(mm)
\]
\[
\Rightarrow \text{Không cần dịch chỉnh răng, nên } d_{w1} = d_{w2} = d_2
\]
\subsection{Kiểm nghiệm bánh răng theo độ bền tiếp xúc}
\subsubsection{Xác định các hệ số tải trọng}
Vận tốc vòng của bánh dẫn:
\[
v = \frac{\pi \cdot d_1 \cdot n_1}{60000} = \frac{\pi \cdot 53.33 \cdot 594.26}{60000} = 1.66(m/s)
\]
Theo bảng 6.13 \cite{reference}, ta chọn cặp chính xác bậc 6, như vậy hệ số sai lệch bước răng $g_0 = 38$  \\
Theo bảng 6.15 \cite{reference}, ta có $\delta_H = 0.002$, $\delta_F = 0.006$  \\
Theo bảng 6.14 \cite{reference} ta có hệ số phân bố không đều tải trọng giữa các đôi răng:\\
\[
K_{H\alpha} = 1.04, \quad K_{F\alpha} = 1.13
\]
Vận tốc:
\[
v_H = \delta_H \cdot g_0 \cdot v\sqrt{\frac{a_w}{u_{br}}} = 7.135(m/s)
\]
Hệ số tải trọng động trong vùng ăn khớp:
\[
K_{Hv} = 1 + \frac{v_H \cdot b_w \cdot d_{w1}}{2T_1 \cdot K_{H\beta} \cdot K_{H\alpha}} = 1.181
\]
Như vậy hệ số tải trọng động tiếp xúc $K_H$:
\[
K_H = K_{H\alpha} \cdot K_{H\beta} \cdot K_{Hv} = 1.302
\]

\subsubsection{Xác định các hệ số xét đến hình dạng, vật liệu và trùng khớp}
Với:
\begin{itemize}
    \item $Z_M = 274 \text{ tra từ bảng 6.5 \cite{reference}}$
    \item $Z_H = \sqrt{\frac{2 \cos \beta_b}{\sin(2\cdot\alpha_{tw})}} = \sqrt{\frac{2\cdot cos15.96}{sin(2\cdot 20.84)}} = 1.7$
    \item $\alpha_{tw} = \arctan \left( \frac{tan\alpha_{nw}}{cos_\beta} \right) = \arctan \left( \frac{tan20}{cos17.01} \right) = 20.84^\circ$
    \item $\beta_b = arctan(cos\alpha_t\cdot tan\beta) = 15.96^\circ$
    \item $Z_\epsilon = \sqrt{\frac{1}{\epsilon_\alpha}} = \frac{1}{1.581} = 0.795$
    \item $\epsilon_\alpha = [1.88-3.2\cdot(\frac{1}{z_1}+\frac{1}{z_2})].cos\beta = [1.88 - 3.2\cdot (\frac{1}{17} + \frac{1}{85})]\cdot cos17.01 = $ 

\end{itemize}
\subsection{Kiểm nghiệm bánh răng theo độ bền uốn}

\subsubsection{Hệ số tải trọng động trong vùng ăn khớp}

\[
K_{Fv} = 1 + \frac{v_P \cdot b_w \cdot d_{w1}}{2 \cdot T_1 \cdot K_{FB} \cdot K_{Fa}} = 1.54
\]

Trong đó:

\[
v_P = \delta_F \cdot g_0 \cdot \sqrt{\frac{v_{br}}{u}} = 21.4(m/s)
\]

Hệ số tải trọng uốn:

\[
K_F = K_{Fa} \cdot K_{FB} \cdot K_{Fv} = 1.986
\]

\subsubsection{Hệ số trùng khớp răng}

\[
Y_\epsilon = \frac{1}{\epsilon_\alpha} = 0.632
\]

\subsubsection{Hệ số kể đến độ nghiêng răng}

\[
Y_\beta = 1 - \frac{\beta}{140} = 0.878
\]
Hệ số số răng $Y_{F1}, Y_{F2}$ tra theo số răng tương đương $z_{v1} = \frac{z_1}{\cos^3 \beta}$ từ bảng 6.18 \cite{reference}:

\[
Y_{F1} = 4.08, \quad Y_{F2} = 3.6
\]

\subsubsection{Tính toán ứng suất uốn}

Như vậy ứng suất uốn của bánh dẫn và bánh bị dẫn:

\[
\sigma_{F1} = \frac{2T_1 K_F Y_\epsilon Y_\beta Y_{F1}}{b_w d_{w1} m_n} = 53.75 MPa < [\sigma_{F1}]
\]
\[
\sigma_{F2} = \frac{\sigma_{F1} Y_{F2}}{Y_{F1}} = 60.91 MPa < [\sigma_{F2}]
\]
Vậy thông số hình học của bộ truyền thỏa điều kiện bền uốn.
\cleardoublepage