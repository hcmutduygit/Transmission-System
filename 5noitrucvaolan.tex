\chapter{Tính chọn nối trục và ổ lăn}
\section{Tính chọn nối trục}
Thông số đã biết:
\begin{itemize}
    \item Momen xoắn ban đầu: $T = 600394.168 \, \text{Nmm}$
    \item Lực vòng tác dụng lên xích: $F_t = 1500 \, \text{Nmm}$
    \item Đường kính trục: $d = 48 \, \text{mm}$
\end{itemize}
.... 

\section{Tính toán ổ lăn}
\subsection{Trục I}
\begin{itemize}
    \item Số vòng quay: $n_1 = 396.67 \, (\text{v}/\text{p})$
    \item Thời gian làm việc: $L_h = 33600 \, \text{giờ}$
    \item Đường kính vòng tròn: $d_1 = 30 \, \text{mm}$
    \item Tải trọng tác dụng lên cạc ổ
    \begin{itemize}
        \item Tải trọng hướng tâm tác dụng lên ổ A:
        \[
        F_A^R = \sqrt{R_{Ax}^2 + R_{Ay}^2} = \sqrt{1390.89^2 + 2236.88^2} = 2634.1 \, \text{N}
        \]
        \item Tải trọng hướng tâm tác dụng lên ổ B:
        \[
        F_B^R = \sqrt{R_{Bx}^2 + R_{By}^2} = \sqrt{1984.15^2 + 899.7^2} = 2178.6 \, \text{N}
        \]
    \end{itemize}
    \item Lực doc trục: $F_{a1} = 1118.4 \, \text{N}$
    \item Do có lực doc trục nên ta chọn ổ bi đôi chồng
    \item Với $d_1 = 30 \, \text{mm}$, tra phụ lục bảng P2.12 tài liệu [2], ta chọn sơ bộ ổ đỡ chặn cỡ trung hẹp có các số liệu như sau:
\end{itemize}
\begin{center}
\begin{tabular}{|c|c|c|c|c|c|c|}
    \hline
    \textbf{Ký hiệu} & \textbf{d} & \textbf{D} & \textbf{B} & \textbf{C} & \textbf{$C_0$} & \textbf{$\alpha$} \\
    \hline
    \textbf{46305} & 30 & 72 & 19 & 25.8 kN & 18.7 kN & 26 \\
    \hline
\end{tabular}
\end{center}
\begin{itemize}
    \item Chọn hệ số $e$:
    \begin{itemize}
        \item Ta có kí hiệu ổ là \textbf{46305} và $\alpha = 26^\circ$
        \item Theo bảng 11.4 tài liệu [2] ta chọn: $e = 0.68$
    \end{itemize}

    \item Chọn hệ số $X$, $Y$:
    \begin{itemize}
        \item $V = 1$: tương ứng với vòng trong quay
        \item $S_A = e.F_R^A = 0.68.2634.1 = 1791.2$ N
        \item $S_B = e.F_R^B = 0.68.2178.6 = 1481.4$ N
        \item $K_t = 1$: hệ số xét đến ảnh hưởng của nhiệt độ với tuổi thọ ổ lăn
        \item $K_v = 1$: hệ số xét đến ảnh hưởng của tải tĩnh, không va đập (bảng 11.3 tài liệu [2])
    \end{itemize}

    \item 
    \[
        F_{ta1} = S_B + F_{a1} = 1481.4 + 1118.4 = 2600 \ \text{N}
    \]
    \[
        F_{ta2} = S_A - F_{a1} = 1791.2 - 1118.4 = 672.8 \ \text{N}
    \]

    \item Tra bảng 11.3 tài liệu [2]:
    \[
        \frac{F_{ta1}}{V.F_R^A} = \frac{2600}{2634.1} = 0.98 \Rightarrow e \rightarrow X = 0.41; Y = 0.87
    \]
    \[
        \frac{F_{ta2}}{V.F_R^B} = \frac{672.8}{2178.6} = 0.31 \Rightarrow X = 1; Y = 0
    \]

    \item Tải trọng quy ước
    \begin{itemize}
        \item Tại A:
        \[
            Q_A = (X V F_r + Y F_a).K_t.K_v = (0.41.1.2634.1 + 0.87.1118.4).1.1 = 2.052 \ \text{kN}
        \]

        \item Tại B:
        \[
            Q_B = (X V F_r + Y F_a).K_t.K_v = (1.1.2178.6 + 0).1.1 = 2.178 \ \text{kN}
        \]

        \item Do $Q_A < Q_B$ nên ta tính toán theo ổ B
    \end{itemize}

    \item Thời gian làm việc:
    \[
        L = \frac{60 L_h . n}{10^6} = \frac{60.33600.396.67}{10^6} = 800 \ \text{triệu vòng}
    \]

    \item Tuổi thọ của ổ:
    \[
        L_h = \frac{10^6}{60n} \left( \frac{C}{Q} \right)^m = \frac{10^6}{60.396.67} \left( \frac{25.8}{2.178} \right)^3 = 69840.1 \ \text{h}
    \]

    \item Kiểm tra tải tĩnh:
    
    Với ổ đỡ chặn $\alpha = 26^\circ$ ta có $X_0 = 0.5; Y_0 = 0.28$ (bảng 11.6 tài liệu [1]), ta có:
    \[
        \left\{
        \begin{aligned}
        C_0 &= X_0 . F_r + Y_0 . F_a = 0.5.2178.6 + 0.37.1118.4 = 1503.1 \ \text{N} \\
        Q_0 &= F_r = 2178.6 \ \text{N}
        \end{aligned}
        \right.
    \]

    \[
        \Rightarrow Q_0 < C_0 = 18.7 \ \text{kN} \Rightarrow \text{ổ đảm bảo điều kiện bền tĩnh}
    \]
    
    \item Số vòng quay tới hạn của ổ
    \begin{itemize}
        \item Theo bảng 11.7 tài liệu [2], với ổ bi đỡ chặn bôi trơn bằng mỡ: $[D_{pw}n] = 1.3 \cdot 10^5$
        \item Đường kính tâm con lăn:
        \[
            D_{pw} = \frac{D + d}{2} = \frac{30 + 72}{2} = 51 \ mm
        \]
    \end{itemize}
    
    Do đó:
    \[
        [n] = \frac{1.3 \cdot 10^5}{51} = 2549 \ \text{v/p} > n = 396.67 \ \text{v/p}
    \]
\end{itemize}