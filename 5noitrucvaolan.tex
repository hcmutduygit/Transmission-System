\chapter{Tính toán thiết kế ổ lăn}
\section{Chọn ổ lăn cho trục dẫn}

\noindent Phản lực tại vị trí ổ lăn:
$$ R_B = \sqrt{R_{Bx}^2 + R_{By}^2} = 1933.449 \, N $$
$$ R_D = \sqrt{R_{Dx}^2 + R_{Dy}^2} = 1112.18 \, N $$

\noindent Ta tính toán chọn ổ lăn theo giá trị phản lực tại ổ B. Do $\frac{F_{a1}}{R_B} > 0.3$ nên ta dùng ổ bi đỡ chặn. Chọn sơ bộ ổ bi đỡ chặn cơ trung với kí hiệu là 46038, tiến hành kiểm nghiệm ổ.

\noindent Tải trọng tĩnh của ổ là $C_0 = 39,2 \, (kN)$, do đó $\frac{F_{a1}}{C_0} \approx 0.042$, theo bảng 11.3 [2], chọn $e = 0,68$. Do kết cấu hệ thống mà chỉ có vòng trong của ổ quay, vì vậy $V = 1$. Tỉ số $\frac{F_{a1}}{VR_B} = 0.381 < e$, theo bảng 11.3 [2], ta chọn $X = 1, Y = 0$.

\noindent Các hệ số tải trọng và ảnh hưởng của nhiệt độ chọn bằng 1 do hệ thống làm việc với chế độ tải trọng tĩnh và nhiệt độ hoạt động dưới $100^\circ C$:
$$ K_q = 1, K_T = 1 $$

\noindent Tính toán lại giá trị lực dọc trục cho ổ:
\begin{itemize}
    \item Tại B:
    $$ F_{aB} = -F_{a1} + eR_B = 584,82 \, N $$
    \item Tại D:
    $$ F_{aD} = F_{a1} + eR_D = 1846,21 \, N $$
\end{itemize}

\noindent Tải trọng động quy ước của ổ:
$$ Q_B = (XVR_B + YF_{aB})K_qK_T = 2219,44 \, N $$
$$ Q_D = (XVR_D + YF_{aD})K_qK_T = 1207,60 \, N $$
\noindent Như vậy ta tính toán kiểm nghiệm theo ổ tại B. Chọn thời gian làm việc của ổ là $L_h = 15000$ giờ, thời gian làm việc tính bằng triệu vòng quay là:
$$ L = \frac{60n_2}{10^6} L_h = 534,84 $$

\noindent Tải trọng động tương đương của ổ:
$$ C_d = Q_B \sqrt[3]{L} = 18,02 \, kN < [C_d] = 21,1 \, kN $$

\noindent Vậy ổ bi đỡ chặn 46306 thỏa yêu cầu.
\noindent Phản lực tại vị trí ổ lăn:
$$ R_A = \sqrt{R_{Ax}^2 + R_{Ay}^2} = 1748.54 \, N $$
$$ R_C = \sqrt{R_{Cx}^2 + R_{Cy}^2} = 1534.45 \, N $$

\noindent Do $\frac{F_{a2}}{R_A} > 0.3$ nên ta dùng ổ bi đỡ chặn. Chọn sơ bộ ổ bi đỡ chặn cơ trung với kí hiệu là 46038, tiến hành kiểm nghiệm ổ.

\noindent Tải trọng tĩnh của ổ là $C_0 = 30.7 \, (kN)$, do đó $\frac{F_{a2}}{C_0} = 0.0246$, theo bảng 11.3 [2], chọn $e = 0,68$. Do kết cấu hệ thống mà chỉ có vòng trong của ổ quay, vì vậy $V = 1$. Tỉ số $\frac{F_{a2}}{VR_A} = 0.432 < e$, theo bảng 11.3 [2], ta chọn $X = 1, Y = 0$.

\noindent Các hệ số tải trọng và ảnh hưởng của nhiệt độ chọn bằng 1 do hệ thống làm việc với chế độ tải trọng tĩnh và nhiệt độ hoạt động dưới $100^\circ C$:
$$ K_q = 1, K_T = 1 $$

\noindent Tính toán lại giá trị lực dọc trục cho ổ:
\begin{itemize}
    \item Tại A:
    $$ F_{aA} = -F_{a2} + eR_A = 432.83 \, N $$
    \item Tại C:
    $$ F_{aC} = F_{a2} + eR_C = 1799.604 \, N $$
\end{itemize}

\noindent Tải trọng động quy ước của ổ:
$$ Q_A = (XVR_A + YF_{aA})K_qK_T = 1748.54 \, N $$
$$ Q_C = (XVR_C + YF_{aC})K_qK_T = 1534.45 \, N $$

\noindent Như vậy ta tính toán kiểm nghiệm theo ổ tại A. Chọn thời gian làm việc của ổ là $L_h = 15000$ giờ, thời gian làm việc tính bằng triệu vòng quay là:
$$ L = \frac{60n_2}{10^6} L_h = 106,97 $$

\noindent Tải trọng động tương đương của ổ:
$$ C_a = Q_A \sqrt[3]{L} = 8.3 \, kN < [C_d] = 39,2 \, kN $$

\noindent Vậy ổ bi đỡ chặn 46308 thỏa yêu cầu.

\section{Tính chọn nối trục đàn hồi}
\begin{itemize}
    \item Momen xoắn trục 2: $T_2 = 321366$ Nmm
    \item Đường kính trục 2 tại vị trí nối trục: 36 mm
    \item Tra bảng 16.10a [3], ta chọn được kích thước cơ bản của nối trục vòng đàn hồi như sau:
\end{itemize}
\begin{table}[H]
    \centering
    \begin{tabular}{|c|c|c|c|c|c|c|c|c|c|c|c|c|c|c|c|c|}
    \hline
    \textbf{T, Nm} & \textbf{d} & \textbf{D} & \textbf{d\textsubscript{m}} & \textbf{L} & \textbf{l} & \textbf{d\textsubscript{l}} & \textbf{D\textsubscript{0}} & \textbf{Z} & \textbf{n\textsubscript{max}} & \textbf{B} & \textbf{B\textsubscript{l}} & \textbf{l\textsubscript{l}} & \textbf{D\textsubscript{3}} & \textbf{l\textsubscript{2}} \\
    \hline
    250 & 36 & 140 & 65 & 165 & 110 & 63 & 105 & 6 & 3800 & 5 & 42 & 30 & 28 & 32 \\
    \hline
    \end{tabular}
\end{table}
\begin{itemize}
    \item Tra bảng 16.10b [3], ta chọn được kích thước cơ bản của vòng đàn hồi:
\end{itemize}
\begin{table}[h]
    \centering
    \begin{tabular}{|c|c|c|c|c|c|c|c|c|}
    \hline
    \textbf{T, Nm} & \textbf{d\textsubscript{c}} & \textbf{d\textsubscript{1}} & \textbf{D\textsubscript{2}} & \textbf{l} & \textbf{l\textsubscript{1}} & \textbf{l\textsubscript{2}} & \textbf{l\textsubscript{3}} & \textbf{h} \\
    \hline
    250 & 14 & M10 & 20 & 62 & 34 & 15 & 28 & 1.5 \\
    \hline
    \end{tabular}
\end{table}
\begin{itemize}
    \item \textbf{Kiểm nghiệm sức bền đập:}
    \[
    \sigma_a = \frac{2kT}{ZD_0 d_c l_3} = \frac{2 \cdot 1 \cdot 321366}{6 \cdot 105 \cdot 14 \cdot 28} = 2.6 \, \text{MPa} < [\sigma_a]
    \]
    trong đó \( k = 1 \): hệ số chê độ làm việc
    \[
    [\sigma_a] = 2 \div 4 \, \text{MPa} : \text{ứng suất đáp cho phép của vòng cao su}
    \]

    \item[\(\rightarrow\)] \textbf{Nối trục thỏa sức bền đập}

    \item \textbf{Kiểm tra sức bền của chốt:}
    \[
    \sigma_u = \frac{k \cdot T \cdot l_0}{0,1 \cdot d^3_c \cdot D_0 \cdot Z} = \frac{1 \cdot 321366 \cdot 41.5}{0.1 \cdot 14^3 \cdot 105 \cdot 6} = 77.15 \, \text{MPa} < [\sigma_u]
    \]
    trong đó \( l_0 = l_1 + \frac{l_2}{2} = 34 + \frac{15}{2} = 41.5 \)
    \[
    [\sigma_u] = 60 \div 80 \, \text{MPa} : \text{ứng suất cho phép của chốt}
    \]
    \item[\(\rightarrow\)] \textbf{Chốt thỏa điều kiện bền}
\end{itemize}