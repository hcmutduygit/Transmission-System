\chapter{Dung sai và lắp ghép}
Căn cứ vào các yêu cầu làm việc của từng chi tiết trong hộp giảm tốc, ta chọn các
kiểu lắp ghép sau:
\section{Dung sai ổ lăn}
Vòng trong ổ lăn chịu tải tuần hoàn, ta lắp ghép theo hệ thống trục lắp trung gian để
vòng ổ không trượt trên bề mặt trục khi làm việc. Do đó, ta phải chọn mối lắp k6, lắp
trung gian có độ dôi, tạo điều kiện mòn đều ổ (trong quá trình làm việc nó sẽ quay làm
mòn đều).
Vòng ngoài của ổ lăn không quay nên chịu tải cục bộ, ta lắp theo hệ thống lỗ. Để ổ có
thể di chuển dọc trục khi nhiệt đô tăng trong quá trình làm việc, ta chọn kiểu lắp trung
gian H7.
\section{Lắp ghép bánh răng trên trục}
Bánh răng lắp lên trục chịu tải vừa, tải trọng tĩnh, va đập nhẹ, ta chọn kiểu lắp ghép
H7/k6.
\section{Lắp ghép vòng chắn dầu trên trục}
Để dễ dàng cho tháo lắp, ta chọn kiểu lắp trung gian H7/Js6
\section{Lắp ghép then}
Theo chiều rộng, chọn kiểu lắp trên trục và kiểu lắp trên bạc là Js9/h9.
Theo chiều cao, sai lệch giới hạn kích thước then là h11.
Theo chiều dài, sai lệch giới hạn kích thước then là h14.
