\chapter*{Lời nói đầu}


\hspace{0.5cm}Trong thực tiễn đời sống và sản xuất, hệ thống truyền động xuất hiện phổ biến ở nhiều lĩnh vực, từ các thiết bị sinh hoạt hằng ngày đến các dây chuyền công nghiệp hiện đại. Hệ thống truyền động đóng vai trò quan trọng trong việc đảm bảo hoạt động ổn định, chính xác và hiệu quả của máy móc thiết bị. Do đó, việc nghiên cứu và thiết kế hệ thống truyền động là một trong những nội dung thiết yếu trong chương trình đào tạo kỹ sư các lĩnh vực có liên quan đến cơ khí. \\

Đồ án hệ thống truyền động là học phần nền tảng thuộc chương trình đào tạo ngành Cơ điện tử, nhằm giúp sinh viên vận dụng kiến thức đã học vào quá trình thiết kế thực tế. Thông qua đồ án, sinh viên không chỉ được tiếp cận quy trình thiết kế hệ thống truyền động một cách bài bản từ phân tích nhiệm vụ, tính toán thiết kế đến thể hiện bản vẽ kỹ thuật mà còn có cơ hội rèn luyện các kỹ năng sử dụng phần mềm chuyên dụng (AutoCAD, AutoCAD Mechanical, Autodesk Inventor...), kết hợp với các kiến thức nền từ các môn học như Nguyên lý máy, Chi tiết máy, Dung sai và kỹ thuật đo,… 
Đây là bước chuẩn bị quan trọng giúp sinh viên hình dung rõ hơn công việc của một kỹ sư thiết kế trong tương lai, từ đó định hướng đúng đắn con đường học tập và phát triển nghề nghiệp của bản thân.\\

Trong quá trình thực hiện đồ án, nhóm chúng em xin chân thành cảm ơn thầy Phạm Minh Tuấn đã tận tình hướng dẫn, hỗ trợ nhóm chúng em về chuyên môn cũng như phương pháp tiếp cận vấn đề một cách hợp lý và khoa học. Đồng thời, nhóm chúng em cũng xin gửi lời cảm ơn sâu sắc đến quý thầy cô trong Bộ môn Thiết kế máy đã truyền đạt những kiến thức quý báu trong suốt thời gian học tập tại trường, tạo nền tảng vững chắc để nhóm chúng em hoàn thành tốt đồ án này.\\

Mặc dù nhóm đã nỗ lực hết mình, song do hạn chế về kiến thức và kinh nghiệm thực tế, bản đồ án chắc chắn không tránh khỏi những thiếu sót. Kính mong thầy cô xem xét và góp ý để nhóm chúng em có thể rút kinh nghiệm, hoàn thiện hơn trong các công việc học tập và nghiên cứu sau này.
